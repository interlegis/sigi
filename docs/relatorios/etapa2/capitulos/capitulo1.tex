%
%  This is part of the SIGI.
%  Copyright (C) 2008 Interlegis
%  See the file relatorio.tex for copying conditions.
%

\section{Informações gerais}
\label{sec:info}

\subsection{SIGI}

O \textbf{SIGI} é um projeto para um Sistema de Informações Gerenciais
do \emph{Interlegis}, escrito na linguagem de programação Python com o
framework para desenvolvimento web Django.

\begin{description}
\item[Página do projeto:]
  http://colab.interlegis.gov.br/wiki/ProjetoSigi
\item[Repositório de código (Subversion):]
  http://repositorio.interlegis.gov.br/SIGI
\end{description}


\subsection{Características}
Lista das principais características do SIGI:

\begin{itemize}
\item Serviço web cliente/servidor, podendo ser disponibilizado tanto
  na internet quanto na intranet;
  
\item Multi-plataforma;
  
\item Baseia-se na interface de administração nativa do Django\\
  (\verb|django.contrib.admin|);
  
\item Gerencia convênios, equipamentos e inventários, serviços
  prestados e composição de Mesas Diretoras das Casas Legislativas;

\item Autenticação no sistema baseada em usuários e grupos, com perfis
  diferentes;

\item Geração de relatórios.
\end{itemize}

\subsection{Licença de Uso}
O SIGI é disponibilizado como \emph{software livre}, isto significa
que você pode redistribuí-lo e/ou modifica-lo dentro dos termos da
Licença Pública Geral GNU (GPL) como publicada pela Fundação do
Software Livre (FSF); na versão 3 da Licença, ou (na sua opinião) em
qualquer versão mais recente.

%
% Local variables:
%   mode: flyspell
%   TeX-master: "relatorio.tex"
% End:
%
