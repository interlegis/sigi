% $Id$
% ---------------------------------------------------------------------------
%
%  This is part of the SIGI.
%  Copyright (C) 2008 Interlegis
%  See the file relatorio.tex for copying conditions.
%

\section{Introdução}
Este documento detalha as atividades desenvolvidas durante a terceira
etapa do projeto de desenvolvimento do Sistema de Informações
Gerenciais do Interlegis (SIGI).

A terceira etapa consiste na documentação dos processos de instalação,
configuração e implantação do SIGI em um servidor GNU/Linux.

Na seção \ref{sec:a1} está disponível o arquivo \verb|LEIA-ME|.

Na seção \ref{sec:a2} está anexado o arquivo \verb|visaogeral.txt|, o
qual apresenta uma visão geral do sistema.

A seção \ref{sec:a3} tem anexado a documentação dos processos de
instalação e configuração.

No Colab, portal colaborativo para a gerência dos projetos de software
do Interlegis, foi publicada a página do projeto SIGI, que pode
ser acessada através do link
\href{http://colab.interlegis.gov.br/wiki/ProjetoSigi}{http://colab.interlegis.g ov.br/wiki/ProjetoSigi}.

\subsection{Terminologia}
\begin{description}
\item[Sistema Gerenciador de Bancos de Dados (SGDB):] conjunto de
  programas de computador (softwares) responsáveis pelo gerenciamento
  de uma base de dados;
\item[Servidor web:] um programa de computador responsável por servir
  dados e páginas web, tais como documentos HTML com objetos embutidos
  (imagens, etc.);
\item[Django:] um framework de desenvolvimento web de alto nível
  escrito em linguagem Python e que estimula o desenvolvimento rápido
  e limpo.
\end{description}

%
% Local variables:
%   mode: flyspell
%   TeX-master: "relatorio.tex"
% End:
%
