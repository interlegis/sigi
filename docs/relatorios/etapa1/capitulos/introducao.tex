% $Id$
% ---------------------------------------------------------------------------
%
%  This is part of the SIGI.
%  Copyright (C) 2008 Interlegis
%  See the file relatorio.tex for copying conditions.
%

\section{Introdução}
O \emph{Interlegis} foi criado para apoiar o processo de modernização
do Poder Legislativo Brasileiro, buscando melhorar a comunicação e o
fluxo de informação entre os legisladores, aumentar a eficiência e
competência das Casas Legislativas, e promover a participação dos
cidadãos nos processos legislativos.

Um \emph{Sistema de Informações Gerenciais} é essencial para o
Programa Interlegis, pois ajuda na tomada de decisões e mantêm um
controle das atividades com as Casas Legislativas.

Atualmente, o Interlegis conta com um Sistema de Informações
Gerenciais limitado tanto tecnologicamente quanto operacionalmente,
por isso a necessidade de criação de um novo sistema atendendo as
demandas atuais e futuras do Programa.

Este documento tem o propósito de detalhar as atividades desenvolvidas
durante a primeira etapa do projeto de desenvolvimento do novo Sistema
de Informações Gerenciais do Interlegis (SIGI), tal como também
descrever as atividades futuras do projeto.

A Seção \ref{sec:visaogeral} contém a \emph{visão geral} do sistema a
ser desenvolvido e as principais etapas para a conclusão do sistema.

Na Seção \ref{sec:requisitos}, é apresentado os requisitos do sistema,
suas principais funcionalidades e características, os requisitos de
\textit{software} e \textit{hardware}, de documentação, performance e
licenciamento.

A Seção \ref{sec:casos} descreve a utilização do sistema através de
\emph{Casos de Uso}, representando os principais atores e suas
interações com o sistema.

Nas Seções \ref{sec:modelo} e \ref{sec:esquema} são apresentados,
respectivamente, o modelo de dados, com suas estruturas e
relacionamentos entre as diversas partes do sistema, e o esquema do
banco de dados.

Na Seção \ref{sec:prototipo}, um protótipo de interface gráfica é
proposto e discriminado.

\subsection{Terminologia}
\begin{description}
\item[Bancos de Dados:] conjunto de informações relacionadas entre
  si, referentes a um mesmo assunto e organizadas de maneira útil, com
  o propósito de servir de base para que o usuário recupere
  informações, tire conclusões e tome decisões.
\item[Sistema de Informação Gerencial (SIG):] agrupa e sintetiza os
  dados de uma organização, transformando-os em informações gerenciais
  que servirão de auxílio na tomada de decisões pelos gestores da
  organização.
\item[Framework:] no desenvolvimento de \textit{software}, um
  \textbf{framework} (ou arcabouço) é uma estrutura de suporte
  definida em que um outro projeto de software pode ser organizado e
  desenvolvido.

  Frameworks são projetados com a intenção de facilitar o
  desenvolvimento de software, habilitando designers e programadores a
  gastarem mais tempo determinando as exigências do software do que
  com detalhes de baixo nível do sistema.
\item[Django:] um framework de desenvolvimento web de alto nível
  escrito em Python e que estimula o desenvolvimento rápido e limpo.
\end{description}

%
% Local variables:
%   mode: flyspell
%   TeX-master: "relatorio.tex"
% End:
%
